%% appendix.tex
%%

%% ==============================
%\chapter{Appendix}
%\label{ch:Appendix}
%% ==============================

\appendix

\addchap{Anhang}

\section{Kernanforderungen an eine mobile Anwendung zur Dokumentation �rztlich �bertragener T�tigkeiten}
\label{anh:anforderungen}

\begin{itemize}
  \item Wie wurden die Anforderungen erhoben?
	  \begin{itemize}
	  		\item Richtlinie
	  		\item Gespr�che mit Raymond
	  		\item Gespr�che Corinna
	  		\item Gesunder Menschenverstand
	  		\item Wissen �ber Zielgruppe
	  		\item Artikel �rzteblatt
	  		\item Diplomarbeit Christina Hardt
	  		\item Projekt VitaBit (Schl�sselnummer, Tourenplan)
	  		\item \ldots   
  	  \end{itemize}
  \item Kernanforderungen (Prio A)
  	\begin{itemize}
	  		\item Funktionalit�t
	  			\begin{itemize}
			  		\item Informationen �ber Klienten m�ssen verf�gbar sein (Patientenakte) [CHECK]
			  		\item Pers�nliche Daten $\rightarrow$ Schutz durch Login-Funktion [CHECK]
			  		\item Dokumentation von T�tigkeiten aus Richtlinie muss m�glich sein [CHECK]
			  		\item ambulante und station�re Pflege abdecken! [CHECK]			  		
			  		\item Dokumentierte T�tigkeiten m�ssen an zentrales System �bertragbar sein [CHECK] 
			  		\item Ablaufplan jeweils f�r eine MFP pro Tag/Schicht	[CHECK]	
			  		\item �bersicht, ob jeweilige Station schon vollst�ndig abgearbeitet [CHECK]	  		
			  		\item Es muss eine T�tigkeit als \textit{nicht durchgef�hrt} markiert werden k�nnen + Begr�ndung,falls etwas			  		
			  		dazwischenkommt [CHECK]
			  		\item Einmal dokumentierte Werte sollten korrigierbar sein [CHECK]
			  		\item Nachvollziehbarkeit von �nderungen [CHECK]		  		
			  		\item Daten "`offline"' auf dem Endger�t verf�gbar, d.h. keine zwangsl�ufige Internetverbindung [CHECK]
			  		\item Dokumentation von Wunden (Fotos) [CHECK]
			  		\item Erfassung von Vitalwerten (Blutdruck, Puls), evtl. mittels automatischer Erfassung [CHECK]
			  		\item Schl�sselnummer f�r Hausbesuche bzw. Zimmernummer bei station�ren Behandlungen (aus VitaBIT)	[CHECK]		  		
			  		\item Klar und �bersichtlich strukturierte Anwendung, intuitive Bedienung (am besten Touch, Endnutzerwissen)
			  		[CHECK]
			  		\item F�r ambulanten Dienst eventuell Navigation mittels GPS [CHECK]
			  		\item Telefongespr�che? [CHECK]
			  		\item \ldots   
		  	    \end{itemize}
		  	\item Abgeleitete technische Anforderungen
			   \begin{itemize}
			     	\item sozialer Kontext $\rightarrow$ g�nstige Hardware [CHECK]			  		
			  		\item \textbf{Hardware:} mobiles Endger�t muss internetf�hig sein (am besten �ber Mobilfunknetze f�r ambulanten
			  		Dienst), eine Kamera haben (Wunddokumentation), am besten Touch-Bedienung, darf trotzdem nicht so teuer sein, evtl. GPS, falls
			  		automatische Erfassung evtl. Bluetooth $\rightarrow$ Android-Tablet [CHECK]
			  		\item \ldots   
		  	  \end{itemize} 
	  		\item \ldots   
  	  \end{itemize}
  \item W�nschenswerte Eigenschaften, um Nutzungskomfort zu steigern (Prio B) [NEIN]
  \item Anforderungen, die �ber die reine Dokumentation oder dessen Unterst�tzung hinausgehen (Prio C, werden in der
  Arbeit nicht behandelt) [NEIN]
\end{itemize}
		
\setcounter{figure}{0}


\dots



