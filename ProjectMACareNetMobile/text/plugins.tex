\chapter{Basisanwendung und Plug-Ins zur Realisierung der identifizierten Anforderungen}
\label{ch:plugins}

Bis hier hin wurde Plug-In-Architektur vorgestellt. Jetzt muss konkret beschrieben werden, welche
Eigenschaften der Gesamtarchitektur und der Plug-Ins welche Anforderungen erf�llen. Hierzu werden
zun�chst die Aufgaben der Basisanwendungen beschrieben (Bereitstellen einer grafischen Struktur,
eines Systems zur Einbindung von Plug-Ins, Herstellen von Sicherheit mittels Log-In und Verwaltung
der Kommunikation mit einem zentralen Server). Anschlie�end werden die wichtigsten Plug-Ins im
Details beschrieben.

\section{Aufgaben und Struktur der Basisanwendung}
\ldots

\section{Einbinden von Plug-Ins in die Basisanwendung}
\ldots

\begin{itemize}
  \item Namenskonventionen
  \item Plug-Struktur (Dateien)
  \item Vorgang des Einbindens
  \item Probleme, Erweiterungsm�glichkeiten
\end{itemize}

\section{Das Plug-In "`Kontakte"'}
F�r alle Plug-Ins feste Beschreibungsstruktur:
\begin{itemize}
  \item Begr�ndung f�r Plug-In (aus welchen Anforderungen geht Plug-In hervor)
  \item Aufbau/Navigationsstruktur
  \item Zentrale Frage, die beantwortet werden muss: Welche Funktionalit�t geht aus welcher
  Anforderung hervor?
\end{itemize}

\ldots

\section{Das Plug-In "`Touren"'}
\ldots

\section{Das Plug-In "`Hilfe"'}
\ldots