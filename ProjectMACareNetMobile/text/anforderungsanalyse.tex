\chapter{Analyse der Anforderungen}
\label{ch:anforderungsanalyse}

\begin{itemize}
  \item Wie wurden die Anforderungen erhoben?
  \item Welche Quellen gab es hierf�r?
  	\begin{itemize}
  		\item Richtlinie, die Delegation erlaubt (f�r grunds�tzliche T�tigkeiten)
  		\item Gespr�ch mit Mitarbeiter eines Pflegedienstes (Raymond)
  		\item Anforderungen aus dem Projekt Vitabit (z.B. Schl�sselnummer)
  		\item Abgeleitete Anforderungen aus Wissen �ber Endnutzer (besonders Usability)
  		\item \ldots   
  	\end{itemize}
  \item Kernanforderungen (Prio A)
  \item W�nschenswerte Eigenschaften, um Nutzungskomfort zu steigern (Prio B)
  \item Anforderungen, die �ber die reine Dokumentation oder dessen Unterst�tzung hinausgehen (Prio C, werden in der Arbeit nicht behandelt)
\end{itemize}