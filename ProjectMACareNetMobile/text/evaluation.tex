%% evaluation.tex
%%

%% ==================
\chapter{Evaluation}
\label{ch:Evaluation}
%% ==================

\section{Evaluation von Nutzbarkeit, Funktionsumfang und Akzeptanz anhand einer Zielgruppenschulung}
\label{sec:evalEndnutzer}

Evaluation der antizipierten Anforderungen aufgrund von Literaturrecherche und logischem Denken. Wichtiger Teil:
Ergebnisse der Schulung der Alten- und Pflegekr�fte vom 19.01.12 . Wie fanden die Teilnehmer die aufgrund der
antizipierten Anforderungen entworfene App? Welche �nderungen wurden vorgeschlagen? Welche Erweiterungen sind
notwendig? $\langle$ Nutzbarkeit, Funktionsumfang, Akzeptanz $\rangle$

\subsection{Vorwissen der Teilnehmer und vorbereitende Ma�nahmen}

\begin{itemize}
  \item Eindruck, Vorwissen (inkl. CAS-Bogen)
  \item CareCM-Schulung (nur erw�hnen, kein Mehrwert)
  \item Grundlagenvermittlung Tablet-PCs
\end{itemize}
 

\subsection{Ermittlung des Status Quo - Ablauf und Schlussfolgerungen}

[Orientierung an Fazit]

\subsection{Anforderungen an CareNet\textit{mobile} aus Sicht der Zielnutzergruppe}
		   
\section{Evaluation der technischen Umsetzung der mobilen Anwendung}

Evaluation der Anforderungen dahingehend, welche umgesetzt werden konnten und in welcher
		   Qualit�t (inkl. der, die noch in der Schulung dazukamen). Vergleich mit
		   Software-Entwicklungs-Standards (ISO xxx), Architekturprinzipien.
Evaluation in zwei Teilen:

\section{Fallstudie Pneunomie}

�berpr�fung der Anwendbarkeit der mobilen Anwendung auf einen Ablauf in der Praxis.