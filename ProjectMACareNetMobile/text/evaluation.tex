%% evaluation.tex
%%

%% ==================
\chapter{Evaluation}
\label{ch:Evaluation}
%% ==================
Evaluation in zwei Teilen:

\begin{enumerate}
	\item \textbf{Teil 1} \\
		   Evaluation der antizipierten Anforderungen aufgrund von Literaturrecherche und logischem
		   Denken. Wichtiger Teil: Ergebnisse der Schulung der Alten- und Pflegekr�fte vom 19.01.12 . Wie
		   fanden die Teilnehmer die aufgrund der antizipierten Anforderungen entworfene App? Welche
		   �nderungen wurden vorgeschlagen? Welche Erweiterungen sind notwendig? $\langle$ Nutzbarkeit,
		   Funktionsumfang, Akzeptanz $\rangle$
	\item \textbf{Teil 2}
		   Evaluation der Anforderungen dahingehend, welche umgesetzt werden konnten und in welcher
		   Qualit�t (inkl. der, die noch in der Schulung dazukamen). Vergleich mit
		   Software-Entwicklungs-Standards (ISO xxx), Architekturprinzipien.
	\item \textbf{Teil 3}
		   Anwendung der App auf einen vereinfachten Fall aus der Praxis 	   
	
\end{enumerate}