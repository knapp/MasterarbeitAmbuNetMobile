\documentclass{thesisclass}

%% -------------------------------
%% |    IM Thesis Template       |
%% -------------------------------
%% Additions by: David Dauer, IM, 2011
%% dauer@iism.uni-karlsruhe.de

%% Notes:
%% Language switch after \begin{document}

% Based on thesisclass.cls of Timo Rohrberg, 2009
% ----------------------------------------------------------------
% Thesis - Main document
% ----------------------------------------------------------------

%% ---------------------------------
%% |      Additional packages      |
%% ---------------------------------
%% 

\usepackage{graphicx}
\usepackage[printonlyused]{acronym}
%http://en.wikibooks.org/wiki/LaTeX/Importing_Graphics#Graphics_storage
\DeclareGraphicsExtensions{.pdf,.png,.jpg}
\graphicspath{{./figures/}} %Use curly braces for each path to add and don't
% forget trailing slash '/'
% \usepackage{epstopdf} %Nice to automatically convert eps figures to pdf
% format  (from inkscape, etc)


%% ---------------------------------
%% | Information about the thesis  |
%% ---------------------------------

\newcommand{\mytype}{\iflanguage{english}{Master Thesis}{Master-Arbeit}} % (Seminar|Bachelor|Master) Thesis
\newcommand{\myname}{Thomas Knapp}
\newcommand{\mytitle}{\iflanguage{english}{Title}{Entwicklung eines mobilen Software-Assistenten zur Unterst�tzung der vernetzten Pflege}}
\newcommand{\myinstitute}{\iflanguage{english}
{Institute of Information Systems and Management (IISM) \\
Information \& Market Engineering}
{Institut f�r Informationswirtschaft und -management (IISM) \\
Information \& Market Engineering}}

\newcommand{\reviewerone}{Prof. Dr. rer. pol. Christof Weinhardt}
\newcommand{\reviewertwo}{Dr. Henner Gimpel}
\newcommand{\advisor}{Bruno Rosales Saurer}
\newcommand{\advisortwo}{Mathias Schmon}
\newcommand{\advisorthree}{Imanol Bernabeu}

\newcommand{\timestart}{01. Oktober 2012}
\newcommand{\timeend}{31. M�rz 2012}
\newcommand{\submissiontime}{31. 03. 2012}

%% -------------------------------
%% |  Information for PDF file   |
%% -------------------------------
%% IM: Auto-Fill this information
\hypersetup{
 pdfauthor={\myname},
 pdftitle={\mytitle},
 pdfsubject={\mytype},
 pdfkeywords={\mytype}
}

%% ---------------------------------
%% | ToDo Marker - only for draft! |
%% ---------------------------------
% Remove this section for final version!
\setlength{\marginparwidth}{20mm}

\newcommand{\margtodo}
{\marginpar{\textbf{\textcolor{red}{ToDo}}}{}}

\newcommand{\todo}[1]
{{\textbf{\textcolor{red}{(\margtodo{}#1)}}}{}}


%% --------------------------------
%% | Old Marker - only for draft! |
%% --------------------------------
% Remove this section for final version!
\newenvironment{deprecated}
{\begin{color}{gray}}
{\end{color}}


%% --------------------------------
%% | Settings for word separation |
%% --------------------------------
% Help for separation:
% In german package the following hints are additionally available:
% "- = Additional separation
% "| = Suppress ligation and possible separation (e.g. Schaf"|fell)
% "~ = Hyphenation without separation (e.g. bergauf und "~ab)
% "= = Hyphenation with separation before and after
% "" = Separation without a hyphenation (e.g. und/""oder)

% Describe separation hints here:
\hyphenation{
% Pro-to-koll-in-stan-zen
% Ma-na-ge-ment  Netz-werk-ele-men-ten
% Netz-werk Netz-werk-re-ser-vie-rung
% Netz-werk-adap-ter Fein-ju-stier-ung
% Da-ten-strom-spe-zi-fi-ka-tion Pa-ket-rumpf
% Kon-troll-in-stanz
}


%% ------------------------
%% |    Including files   |
%% ------------------------
% Only files listed here will be included!
% Userful command for partially translating the document (for bug-fixing e.g.)
\includeonly{%
titlepage,
text/acronyms,
text/einleitung,
text/zieleArbeit,
text/anforderungsanalyse,
text/plugins,
text/evaluation,
text/ausblick,
text/fazit,
text/declaration,
text/appendix
}

%%%%%%%%%%%%%%%%%%%%%%%%%%%%%%%%%
%% Here, main documents begins %%
%%%%%%%%%%%%%%%%%%%%%%%%%%%%%%%%%
\begin{document}

% Comment the following line out for German text
% \selectlanguage{english}

\frontmatter
\pagenumbering{roman}
\include{titlepage}
% IM Style: No additional blank page
% \blankpage


%% -------------------
%% |   Directories   |
%% -------------------
\tableofcontents
\listoffigures
\listoftables
\include{text/acronyms}
% IM Style: No additional blank page
% \blankpage


%% -----------------
%% |   Main part   |
%% -----------------
\mainmatter
\pagenumbering{arabic}
%% introduction.tex
%%

%% ==============================
\chapter{Einleitung}
\label{ch:einleitung}
%% ==============================
\begin{itemize}
\item Kontext mit �rztemangel 
\item Als L�sungsansatz Delegation von Leistungen an qualitfiziertes Pflegepersonal ( wie werden MFPs ausgebildet???)
\item Effizienzsteigerung
\item Einbringen von Statistiken �ber Bev�lkerungsentwicklung, Demographiewandel
\end{itemize}

\dots


\chapter{Ziele der Arbeit}
\label{ch:zielederarbeit}


\begin{itemize}
  \item Ganz allgemein: Mobile, elektronische Dokumentation delegierter �rztlicher T�tigkeiten durch Angeh�rige der Alten- und Pflegeberufe.
  \item Entwicklung einer App zun�chst f�r Android inkl. Serverkommunikation (Aufsetzen eines erweiterbaren Middleware-Ansatzes)
  \item Anpassung einer SaaS-L�sung, um den Anforderungen von CareNetmobile zu gen�gen (CareNet)
  \item Durchf�hrung einer Anforderungsanalyse (Sammeln von Anforderungen, Priorisierung, Absch�tzung was m�glich ist)
  \item Implementierung der Kernanforderungen, um eine grunds�tzliche Dokumentation zu erm�glichen
  \item Evaluation in mehreren Schritten:
  \begin{itemize}
    \item Messen der Architektur an ISO-Software-Qualit�tskriterien (vor allem Wartbarkeit) (Focus: Technische Umsetzung)
    \item Schulung/Anforderungsevaluation mit Endnutzern (Pflegepersonal) (Focus: Usability/ Akzeptanztests)
    \item �berpr�fung, ob ein bestimmter Prozess (z.B. Pneunomie) mit der App abgebildet werden kann
  \end{itemize}  
\end{itemize}
\chapter{Analyse der Anforderungen}
\label{ch:anforderungsanalyse}

\begin{itemize}
  \item Wie wurden die Anforderungen erhoben?
  \item Welche Quellen gab es hierf�r?
  	\begin{itemize}
  		\item Richtlinie, die Delegation erlaubt (f�r grunds�tzliche T�tigkeiten)
  		\item Gespr�ch mit Mitarbeiter eines Pflegedienstes (Raymond)
  		\item Anforderungen aus dem Projekt Vitabit (z.B. Schl�sselnummer)
  		\item Abgeleitete Anforderungen aus Wissen �ber Endnutzer (besonders Usability)
  		\item \ldots   
  	\end{itemize}
  \item Kernanforderungen (Prio A)
  \item W�nschenswerte Eigenschaften, um Nutzungskomfort zu steigern (Prio B)
  \item Anforderungen, die �ber die reine Dokumentation oder dessen Unterst�tzung hinausgehen (Prio C, werden in der Arbeit nicht behandelt)
\end{itemize}
\chapter{Basisanwendung und Plug-Ins zur Realisierung der identifizierten Anforderungen}
\label{ch:pluginarchitektur}

Bis hier hin wurde Plug-In-Architektur vorgestellt. Jetzt muss konkret beschrieben werden, welche
Eigenschaften der Gesamtarchitektur und der Plug-Ins welche Anforderungen erf�llen. Hierzu werden
zun�chst die Aufgaben der Basisanwendungen beschrieben (Bereitstellen einer grafischen Struktur,
eines Systems zur Einbindung von Plug-Ins, Herstellen von Sicherheit mittels Log-In und Verwaltung
der Kommunikation mit einem zentralen Server). Anschlie�end werden die wichtigsten Plug-Ins im
Details beschrieben.

\section{Aufgaben und Struktur der Basisanwendung}
\ldots

\section{Einbinden von Plug-Ins in die Basisanwendung}
\ldots

\begin{itemize}
  \item Namenskonventionen
  \item Plug-Struktur (Dateien)
  \item Vorgang des Einbindens
  \item Probleme, Erweiterungsm�glichkeiten
\end{itemize}

\section{Das Plug-In "`Kontakte"'}
F�r alle Plug-Ins feste Beschreibungsstruktur:
\begin{itemize}
  \item Begr�ndung f�r Plug-In (aus welchen Anforderungen geht Plug-In hervor)
  \item Aufbau/Navigationsstruktur
  \item Zentrale Frage, die beantwortet werden muss: Welche Funktionalit�t geht aus welcher
  Anforderung hervor?
\end{itemize}

\ldots

\section{Das Plug-In "`Touren"'}
\ldots

\section{Das Plug-In "`Hilfe"'}
\ldots
%% evaluation.tex
%%

%% ==================
\chapter{Evaluation}
\label{ch:Evaluation}
%% ==================
Evaluation in zwei Teilen:

\begin{enumerate}
	\item \textbf{Teil 1} \\
		   Evaluation der antizipierten Anforderungen aufgrund von Literaturrecherche und logischem
		   Denken. Wichtiger Teil: Ergebnisse der Schulung der Alten- und Pflegekr�fte vom 19.01.12 . Wie
		   fanden die Teilnehmer die aufgrund der antizipierten Anforderungen entworfene App? Welche
		   �nderungen wurden vorgeschlagen? Welche Erweiterungen sind notwendig? $\langle$ Nutzbarkeit,
		   Funktionsumfang, Akzeptanz $\rangle$
	\item \textbf{Teil 2}
		   Evaluation der Anforderungen dahingehend, welche umgesetzt werden konnten und in welcher
		   Qualit�t (inkl. der, die noch in der Schulung dazukamen). Vergleich mit
		   Software-Entwicklungs-Standards (ISO xxx), Architekturprinzipien.
	\item \textbf{Teil 3}
		   Anwendung der App auf einen vereinfachten Fall aus der Praxis 	   
	
\end{enumerate}
\chapter{Fazit und zuk�nftige Entwicklungs- und Einsatzm�glichkeiten}
\label{ch:ausblick}

Ausblick auf m�gliche Erweiterungen:

\begin{itemize}
  \item Erweiterbarkeit des Ansatzes ausgrund von Plug-In-Struktur herausstellen
  \item Weitere Funktionen, um Komfort zu steigern (Prio C Anforderungen)
  	\begin{itemize}
  	  \item Schlie�en eines Falles beim Entfernen vom Einsatzort (damit keine �berlangen
  	  Verweildauern gespeichert werden)
  	  \item Login zu verschiedenen Instanzen von CareNet (Auswahl beim Login)
  	  \item Unterschiedliche Verf�gbarkeit von Plug-ins in Abh�ngigkeit des Rechten desjenigen, der
  	  sich anmeldet
  	\end{itemize}
\end{itemize}

\begin{enumerate}
  \item Kurzer Vergleich der nativen Entwicklung unter Android mit der hybriden Entwicklung (Wie hat sich die Entwicklung gestaltet?)
  \begin{enumerate}
    \item Typsicherheit
    \item Namensr�ume (z.B. Gefahr von �berschneidung von Variablennamen und Methodennamen)
    \item Unterst�tzung durch native Java-Bibliotheken vs. JavaScript-Bibliotheken
    \item Einfachheit der Implementierung mit HTML/JavaScript
    \item Vorteil der Potierbarkeit (Test w�re hier praktisch)
  \end{enumerate}
\end{enumerate}
%% conclusion.tex
%%

%% ==================
\chapter{Fazit}
\label{ch:fazit}
%% ==================

\begin{enumerate}
  \item Kurzer Vergleich der nativen Entwicklung unter Android mit der hybriden Entwicklung (Wie hat sich die Entwicklung gestaltet?)
  \begin{enumerate}
    \item Typsicherheit
    \item Namensr�ume (z.B. Gefahr von �berschneidung von Variablennamen und Methodennamen)
    \item Unterst�tzung durch native Java-Bibliotheken vs. JavaScript-Bibliotheken
    \item Einfachheit der Implementierung mit HTML/JavaScript
    \item Vorteil der Potierbarkeit (Test w�re hier praktisch)
  \end{enumerate}
\end{enumerate}

Erste Worte des Fazits\dots
Und eine Referenz \cite{kopetsch2010}\\
Und noch eine \cite{stat2050}
 
\include{text/declaration}

%% ----------------
%% |   Appendix   |
%% ----------------
% IM Style: No additional blank page
% \cleardoublepage

\input{text/appendix}


%% --------------------
%% |   Bibliography   |
%% --------------------
\cleardoublepage
\phantomsection
\addcontentsline{toc}{chapter}{Literaturverzeichnis}

% IM Style
%\iflanguage{english}
%\bibliographystyle{chicago}	% english style
%\bibliographystyle{chicagode}	% german style

% Informatik-Style
%\iflanguage{english}
%{\bibliographystyle{IEEEtranSA}}	% english style
% \bibliographystyle{babalpha-fl}	% german style
												  
% Use IEEEtran for numeric references
\bibliographystyle{IEEEtranSA}

\bibliography{bibliography}


\end{document}
